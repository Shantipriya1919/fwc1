\documentclass[10pt]{article}
\usepackage{graphicx}
\usepackage[none]{hyphenat}
\usepackage{listings}
\usepackage[english]{babel}
\usepackage{siunitx}
\usepackage{caption}
\usepackage{booktabs}
\usepackage{array}
\usepackage{extarrows}
\usepackage{enumerate}
\usepackage{enumitem}
\usepackage{amsmath}
\usepackage{commath}
\usepackage{gensymb}
\usepackage{amssymb}
\usepackage{multicol}
\usepackage[utf8]{inputenc}
\lstset{
	frame=single,
	breaklines=true
}
\usepackage{hyperref}
%\usepackage[margin=0.8in]{geometry}
%\usepackage{exsheets}% also loads the `tasks' package
\usepackage{atbegshi}
\AtBeginDocument{\AtBeginShipoutNext{\AtBeginShipoutDiscard}}

%new macro definitions
\newcommand{\mydet}[1]{\ensuremath{\begin{vmatrix}#1\end{vmatrix}}}
\providecommand{\brak}[1]{\ensuremath{\left(#1\right)}}
\newcommand{\solution}{\noindent \textbf{Solution: }}
\newcommand{\myvec}[1]{\ensuremath{\begin{pmatrix}#1\end{pmatrix}}}
\providecommand{\norm}[1]{\left\1Vert#1\right\rVert}
\let\vec\mathbf{}


%\SetEnumitemKey{twocol}{
%	before=\raggedcolumns\begin{multicols}{2},
%	after=\end{multicols}}
%\SetEnumitemKey{fourcol}{
%	before=\raggedcolumns\begin{multicols}{4},
%	after=\end{multicols}}	


\begin{document}
\begin{center}
\title{\textbf{VECTORS}}
\date{\vspace{-5ex}}
\maketitle
\end{center}
\section*{12$^{th}$Math - Chapter 10}
This is Problem-16 from Exercise 10.5\\
\begin{enumerate}
\item If $\theta$ is the angle between two vectors $\overrightarrow{\vec{a}}$ and $\overrightarrow{\vec{b}}$,then$\overrightarrow{\vec{a}}\cdot\overrightarrow{\vec{b}}\ge 0.$
\begin{enumerate}
\item 0$<\theta<\frac{\pi}{2}$
\item 0$\le\theta\le\frac{\pi}{2}$
\item 0$<\theta<\pi$
\item 0$\le\theta\le\pi$
\end{enumerate}
\solution
Given
\begin{align}
\vec{a},\vec{b} \text{ are two vectors }\\
{\vec{a}}^{\top}{\vec{b}}\ge0\\
\text{ Assume that $\vec{a}$,$\vec{b}$ are , }\notag \\
\vec{a}&=\myvec{4\\3}\\
\vec{b}&=\myvec{5\\12}\\
\text{ We know that }\notag \\
\theta&=\cos^{-1}\brak{\frac{{\vec{a}}^{\top}{\vec{b}}}{\norm{\vec{a}}\norm{\vec{b}}}} \\
\implies {\vec{a}}^{\top}{\vec{b}}&=\cos\theta\norm{\vec{a}}\norm{\vec{b}}\\
\norm{\vec{a}}&=\sqrt{{\vec{a}_1}^{2} + {\vec{a}_2}^{2}}
\end{align}
\begin{align}
\text{ Verification: }\notag \\
\norm{\vec{a}}&=\sqrt{{4^2}+{3^2}}\\
\implies &=5\\
\norm{\vec{b}}&=\sqrt{{5^2}+{12^2}}\\
\implies &=13\\
\text{ for $\theta=0$ }\notag \\
{\vec{a}}^{\top}{\vec{b}}&=\cos(0)(5)13)\\
\implies &=65\\
\implies {\vec{a}}^{\top}{\vec{b}}\ge0 \notag \\
\text{ for $\theta=\frac{\pi}{2}$ }\notag \\
{\vec{a}}^{\top}{\vec{b}}&=\cos(\frac{\pi}{2})(5)(13) \\
\implies &=0\\
\implies {\vec{a}}^{\top}{\vec{b}}\ge0 \notag\\
\text{ for $\theta=\pi$ }\notag \\
{\vec{a}}^{\top}{\vec{b}}&=\cos(\pi)(5)(13)\\
\implies &=-65\\
\implies {\vec{a}}^{\top}{\vec{b}}<0 \notag
\end{align}
Therefore, the $\theta$ is $0\le\theta\le\frac{\pi}{2}$ .So, option (b) is correct.
\end{enumerate}
\end{document}
