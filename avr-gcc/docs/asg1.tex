\documentclass[journal,12pt,twocolumn]{IEEEtran}
%
\usepackage{setspace}
\usepackage{textcomp}
\usepackage{gensymb}
\usepackage{xcolor}
\usepackage{caption}
%\usepackage{subcaption}
%\doublespacing
\singlespacing

\usepackage{graphicx}
\graphicspath{{./images/}}
\usepackage[colorlinks=true, urlcolor=blue, linkcolor=black]{hyperref}
%\usepackage[parfill]{parskip}
%\usepackage{amssymb}
%\usepackage{relsize}
\usepackage[cmex10]{amsmath}
\usepackage{mathtools}
\usepackage{amsmath}
%\usepackage{amsthm}
%\interdisplaylinepenalty=2500
%\savesymbol{iint}
%\usepackage{txfonts}
%\restoresymbol{TXF}{iint}
%\usepackage{wasysym}
\usepackage{amsthm}
\usepackage{mathrsfs}
\usepackage{txfonts}
\usepackage{stfloats}
\usepackage{cite}
\usepackage{cases}
\usepackage{subfig}
%\usepackage{xtab}
\usepackage{hyperref}
\usepackage{longtable}
\usepackage{multirow}
%\usepackage{algorithm}
%\usepackage{algpseudocode}
\usepackage{enumitem}
\usepackage{mathtools}
%\usepackage{eenrc}
%\usepackage[framemethod=tikz]{mdframed}
%\usepackage{hyperref}
\usepackage{listings}
    \usepackage[latin1]{inputenc}                                 %%
    \usepackage{color}                                            %%
    \usepackage{array}                                            %%
    \usepackage{longtable}                                        %%
    \usepackage{calc}                                             %%
    \usepackage{multirow}                                         %%
    \usepackage{hhline}                                           %%
    \usepackage{ifthen}                                           %%
  %optionally (for landscape tables embedded in another document): %%
    \usepackage{lscape}     
\usepackage{tikz}
\usepackage{circuitikz}
\usepackage{karnaugh-map}
\usepackage{pgf}

\usepackage{url}
\def\UrlBreaks{\do\/\do-}



%\usepackage{stmaryrd}


%\usepackage{wasysym}
%\newcounter{MYtempeqncnt}
\DeclareMathOperator*{\Res}{Res}
%\renewcommand{\baselinestretch}{2}
\renewcommand\thesection{\arabic{section}}
\renewcommand\thesubsection{\thesection.\arabic{subsection}}
\renewcommand\thesubsubsection{\thesubsection.\arabic{subsubsection}}

\renewcommand\thesectiondis{\arabic{section}}
\renewcommand\thesubsectiondis{\thesectiondis.\arabic{subsection}}
\renewcommand\thesubsubsectiondis{\thesubsectiondis.\arabic{subsubsection}}



%\surroundwithmdframed[width=\columnwidth]{lstlisting}
\def\inputGnumericTable{}                                 %%
\lstset{
%language=C,
frame=single, 
breaklines=true,
columns=fullflexible
}
 

\begin{document}
%

\theoremstyle{definition}
\newtheorem{theorem}{Theorem}[section]
\newtheorem{problem}{Problem}
\newtheorem{proposition}{Proposition}[section]
\newtheorem{lemma}{Lemma}[section]
\newtheorem{corollary}[theorem]{Corollary}
\newtheorem{example}{Example}[section]
\newtheorem{definition}{Definition}[section]
%\newtheorem{algorithm}{Algorithm}[section]
%\newtheorem{cor}{Corollary}
\newcommand{\BEQA}{\begin{eqnarray}}
\newcommand{\EEQA}{\end{eqnarray}}
\newcommand{\define}{\stackrel{\triangle}{=}}
\vspace{2cm}
\title{ 
Assignment
}

\author{M Shantipriya}


\maketitle
\tableofcontents
%
%\newpage
\section*{\textbf{Question}}
\begin{flushleft}
For the given boolean expression f=$\bar{a}$$\bar{b}$$\bar{c}$+$\bar{a}$$b$$\bar{c}$+$a$$\bar{b}$$\bar{c}$+$abc$+$ab$$\bar{c}$ , the minimized Product of Sum (POS) expression is
\end{flushleft}

\begin{flushright}
\textbf{Q(16), GATE/IN-2011}
\end{flushright}
\section{\textbf{Components}}
\input{components.tex}
\begin{table}[!h]
\centering
\caption{}
\label{table:7447_disp}
\end{table}

\begin{figure}[!h]
2.The figure given below is the pin diagram of Seven Segment Display.\\
\begin{center}
\resizebox {0.4\columnwidth} {!} {
\input{sevenseg.tex}
}
\end{center}

\caption{}
\label{fig:sevenseg}
\end{figure}

\begin{table}[!h]
1.The table given below is the connections between 7447 BCD Decoder and Seven Segment Display\\
\centering
\input{7447_disp.tex}
\caption{}
\label{table:7447_disp}
\end{table}
\begin{figure}[!h]
3.The diagram below shows the pin diagram of 7447 BCD Decoder.The output pins of 7447 is connected to Seven Segment Display using Table 2.
\begin{center}
\resizebox {1.2\columnwidth} {!} {
\input{7447.tex}
}
\end{center}
\caption{}
\label{fig:7447}
\end{figure}

\section{\textbf{Truthtable}}

%begin{table}[]
    \begin{center}
    \begin{tabular}{ |c |c |c |c |c |}
\hline
\newline
\textbf{a} & \textbf{b} & \textbf{c} & \textbf{f} \\
\hline
 %Resistor & 220Ohm & 1 \\ 
 0 & 0 & 0 &1 \\  
 0 & 0 & 1 &0 \\ 
 0 & 1 & 0 &1 \\ 
 0 & 1 & 1 &0 \\ 
 1 & 0 & 0 &1 \\ 
 1 & 0 & 1 &0 \\ 
 1 & 1 & 0 &1 \\ 
 1 & 1 & 1 &1 \\ 
 \hline
 \end{tabular}
\label{Truth table}
%\end{table}
\end{center}

\section{\textbf{K-MAP}}
    
    \begin{karnaugh-map}[4][2][1][${\textbf{bc}}$][${\textbf{a}}$]
         \maxterms{1,3,5}
         \minterms{0,2,4,6,7}
         \implicant{1}{5}
         \implicant{1}{3}
\end{karnaugh-map}

The minimized expression is f=(b+$\bar{c}$)(a+$\bar{c}$)


\section{\textbf{Procedure}}
\begin{enumerate}


\item The given boolean expression is f=$\bar{a}$$\bar{b}$$\bar{c}$+$\bar{a}$$b$$\bar{c}$+$a$$\bar{b}$$\bar{c}$+$abc$+$ab$$\bar{c}$\\from this we can write the minimized POS expression as follows

\begin{align*}
f&={\bar{a}}\bar{b}\bar{c}+\bar{a}b\bar{c}+a\bar{b}\bar{c}+abc+ab\bar{c}\\[\parskip]
&=\bar{a}\bar{c}(\bar{b}+b)+a\bar{c}(\bar{b}+b)+abc\\[\parskip]
&=\bar{a}\bar{c}+a\bar{c}+abc \quad\text{(additive identity [$\bar{b}$+b =1])}\\[\parskip]
&=\bar{c}(\bar{a}+a)+abc\\[\parskip]
&=\bar{c}+abc \quad\text{(additive identity [$\bar{a}$+a=1])}\\[\parskip]
&=(\bar{c}+b)(\bar{c}+a)(\bar{c}+c) \quad\text{(distributivelaw A+BC=(A+B)(A+C))}\\[\parskip]
&=(b+\bar{c})(a+\bar{c}) \quad\text{(additive identity [$\bar{c}$+c=1])}
\end{align*}
\item connect the circuit using 7447 BCD-Seven segment display decoder and Arduino.\\
\item connect the seven segment pins to 7447 using Table 2.\\
\item connect the pin A of 7447 to D2 of Arduino and remaining pins B,C and D to GND.\\ \item connect the pins D8,D9,D10 to 0's and 1's.Change the pins simultaneously to verify the POS expression truth table.\\
\item Verify the miinimized POS expression operation in avr-gcc using the following code and making pin connections according to fig 2,Table 2\\
\end{enumerate}
\textbf{Observe the truthtable and verify the program by executing the link provided below.}\\
\begin{center}
\fbox{\parbox{8.5cm}{\url{https://github.com/Shantipriya1919/fwc1}}}
\end{center}
\end{document}
