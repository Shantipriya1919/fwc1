\documentclass[10pt]{article}
\usepackage{graphicx}
\def\inputGnumericTable{}
\usepackage[latin1]{inputenc}
\usepackage{fullpage}
\usepackage{color}
\usepackage{array}
\usepackage{longtable}
\usepackage{calc}
\usepackage{multirow}
\usepackage{hhline}
\usepackage{ifthen}
\usepackage{amsmath}
\usepackage[none]{hyphenat}
\usepackage{listings}
\usepackage[english]{babel}
\usepackage{siunitx}
\usepackage{caption}
\usepackage{booktabs}
\usepackage{array}
\usepackage{extarrows}
\usepackage{enumerate}
\usepackage{enumitem}
\usepackage{amsmath}
\usepackage{commath}
\usepackage{gensymb}
\usepackage{amssymb}
\usepackage{multicol}
%\usepackage[utf8]{inputenc}
\lstset{
 frame=single,
 breaklines=true
}
\usepackage{hyperref}
\usepackage[margin=0.65in]{geometry}	 
%\usepackage{exsheets}% also loads the `tasks' package
\usepackage{atbegshi}
\AtBeginDocument{\AtBeginShipoutNext{\AtBeginShipoutDiscard}}

%new macro definitions
\renewcommand{\labelenumi}{(\roman{enumi})}
\newcommand{\mydet}[1]{\ensuremath{\begin{vmatrix}#1\end{vmatrix}}}
\providecommand{\brak}[1]{\ensuremath{\left(#1\right)}}
\newcommand{\solution}{\noindent \textbf{Solution: }}
\newcommand{\myvec}[1]{\ensuremath{\begin{pmatrix}#1\end{pmatrix}}}
\newenvironment{amatrix}[1]{%
	\left(\begin{array}{@{}*{#1}{c}|c@{}}
}{%
	\end{array}\right)
}

\newcommand{\myaugvec}[2]{\ensuremath{\begin{amatrix}{#1}#2\end{amatrix}}}
\providecommand{\norm}[1]{\left\1Vert#1\right\rVert}
\let\vec\mathbf{}


%\SetEnumitemKey{twocol}{
% before=\raggedcolumns\begin{multicols}{2},
% after=\end{multicols}}
%\SetEnumitemKey{fourcol}{
% before=\raggedcolumns\begin{multicols}{4},
% after=\end{multicols}} 


\begin{document}
\begin{center}
\title{\textbf{TRIANGLES}}
\date{\vspace{-5ex}}
\maketitle
\end{center}
\section*{9$^{th}$Math - Chapter 7}
This is Problem-8 from Exercise 7.1\\\\
In right triangle ABC, right angled at C, M is the mid-point of hypotenuse AB. C is joined to M and produced to a point D such that DM = CM. Point D is joined to point B (see Figure \ref{fig:1}). Show that:
\begin{enumerate}
\item $\triangle AMC \cong \triangle BMD$
\item $\angle DBC$ is a right angle.
\item $\triangle DBC \cong \triangle ACB$
\item $CM = \dfrac{1}{2}AB$
\end{enumerate}

\begin{figure}[!h]
	\begin{center}
		\includegraphics[width=\columnwidth]{./figs/fig.pdf}
	\end{center}
\caption{}
\label{fig:1}
\end{figure}

\section*{\large Construction:}
The input parameters for construction
\begin{table}[h!]
	\small
	\centering
	%\subimport{../tables/}{table.tex}
     \input{tables/table.tex}
%	\caption{}
	\label{table:table}
\end{table}
\begin{align}
	\vec{A}&=\myvec{a\\b}\\
	\vec{B}&=\myvec{0\\0}\\
	\vec{C}&=\myvec{a\\0}\\
	\vec{D}&=\myvec{0\\b}\\
	\vec{M}&=\frac{A+B}{2}=\frac{1}{2}\myvec{a\\b}
\end{align}
\solution
Given
\begin{align}
	\vec{M}&=\frac{A+B}{2}
	\label{eq:1}\\
	\norm{\vec{D}-\vec{M}}&=\norm{\vec{C}-\vec{M}}
	\label{eq:2}\\
	\angle ACB&=90\degree
	\label{eq:3}
\end{align}
\begin{enumerate}
\item $\triangle AMC \cong \triangle BMD$
\begin{align} 
	\text{ from \eqref{eq:1} we can write as, }\\
	\norm{\vec{A}-\vec{M}} &= \norm{\vec{B}-\vec{M}}
	\label{eq:4}\\
	\angle AMC &=\cos^{-1}\brak{\frac{\brak{\vec{A}-\vec{M}}^{\top}\brak{\vec{C}-\vec{M}}}{\norm{\vec{A}-\vec{M}}\norm{\vec{C}-\vec{M}}}}\\
	 \angle DMB &= \cos^{-1}\brak{\frac{\brak{\vec{D}-\vec{M}}^{\top}\brak{\vec{B}-\vec{M}}}{\norm{\vec{D}-\vec{M}}\norm{\vec{B}-\vec{M}}}}\\
	\angle AMC = \angle DMB & \thickapprox 74\degree
	\label{eq:5}
\end{align}
	from \eqref{eq:2}, \eqref{eq:4} and \eqref{eq:5},
\begin{align}
	\triangle AMC &\cong \triangle BMD
	\label{eq:6}\\
	\text{ from \eqref{eq:6} we can say that, }\\
	\norm{\vec{D}-\vec{B}}&=\norm{\vec{A}-\vec{C}}
	\label{eq:7}
\end{align}
\item $\angle DBC$ is a right angle
\begin{align}
	\brak{\vec{D}-\vec{B}}^{\top}\brak{\vec{C}-\vec{B}} &= \myvec{0 & b}\myvec{a\\0}\\
	&=0\\
	\implies BD & \perp BC\\
	\implies  \angle DBC &= 90\degree	 
	\label{eq:8}
\end{align}
\item $\triangle DBC \cong \triangle ACB$
\begin{align}
	\text{ from  \eqref{eq:3} and \eqref{eq:8} }\\
	\angle ACB &= \angle DBC =90\degree 
	\label{eq:9}\\
	\norm{\vec{B}-\vec{C}}&=\norm{\vec{C}-\vec{B}} 
	\label{eq:10}
\end{align}
from \eqref{eq:7}, \eqref{eq:9} and \eqref{eq:10},
\begin{align}
	\triangle DBC &\cong \triangle ACB
	\label{eq:11}
\end{align}
\item $CM = \dfrac{1}{2}AB$
\begin{align}
	\norm{\vec{A}-\vec{B}}&=\norm{\myvec{a\\b}}\\
	\norm{\vec{C}-\vec{D}}&=\norm{\myvec{a\\-b}}\\
	\implies \norm{\vec{A}-\vec{B}} &= \norm{\vec{C}-\vec{D}}\\
	\text{ or, } AB &= CD
	\label{eq:12}	
\end{align}
from \eqref{eq:1} and \eqref{eq:2}, $\vec{M}$ is midpoint of both AB and CD\\
from \eqref{eq:12}
\begin{align}
	CM = \frac{1}{2}CD = \frac{1}{2}AB 
\end{align}
\end{enumerate}
\end{document}
