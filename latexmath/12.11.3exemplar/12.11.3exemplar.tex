\documentclass[12pt]{article}
\usepackage{graphicx}
\usepackage{enumerate}
\usepackage{amsmath}
\usepackage{gensymb}
\usepackage{exsheets}% also loads the `tasks' package


\begin{document}
\title{\textbf{CHAPTER-11 \\THREE DIMENSIONAL GEOMETRY}}
\maketitle
\begin{center}
\end{center}
\section*{EXERCISE-11.3}
\section*{Short Answer(S.A)}
\begin{enumerate}

\item  Find the position vector of a point A in space such that $\overrightarrow{OA}$ is inclined at 60 $^{\circ}$ to OX and at 45 $^{\circ}$ to OY and $|\overrightarrow{OA}| =10$ units.
\item  Find the vector equation of the line which is parallel to the vector $3\hat{i}-2\hat{j}+6\hat{k}$ and which passes through the point $(1,-2,3)$.
\item Show the lines
\begin{align*}
\frac{x-1}{2}=\frac{y-2}{3}=\frac{z-3}{4}
\end{align*}
\begin{align*}
and  \hspace{0.5cm} \frac{x-4}{5}=\frac{y-1}{2}=z \hspace{0.5cm} intersect.
\end{align*}
\\ Also, find their point of intersection.
\item Find the angle between the lines \\\\ $\overrightarrow{r}=3\hat{i}-2\hat{j}+6\hat{k}+\lambda(2\hat{i}+\hat{j}+2\hat{k})$ and $\overrightarrow{r}=(2\hat{j}-5\hat{k})+\mu(6\hat{i}+3\hat{j}+2\hat{k})$
\item Prove that the line through A$(0,-1,-1)$ and B$(4,5,1)$ intersects the line through C$(3,9,4)$ and D$(-4,4,4)$.
\item Prove that the lines $x=py+q , z=ry+s$ and $x=p'y+q', z=r'y+s'$ are perpendicular if $pp'+rr'+1=0$.
\item Find the equation of a plane which  bisects perpendicularly the line joining the points A$(2,3,4)$ and B$(4,5,8)$ at right angles.
\item Find the equation of a plane which is at a distance 3$\sqrt{3}$ units from origin and the normal to which is equally inclined to coordinate axis.
\item If the line drawn from the point $(-2,-1,-3)$ meets a plane at right angle at the point $(1,-3,3)$, find the equation of the plane.
\item Find the equation of the plane through the points $(2,1,0)$, $(3,-2,-2)$ and $(3,1,7)$.
\item Find the equations of the two lines through the origin which intersect the line \(\displaystyle \frac{x-3}{2}=\frac{y-3}{1}=\frac{z}{1}\) at angles of \(\displaystyle \frac{\pi}{3}\)each.
\item Find the angle between the lines whose direction cosines are given by the equations $l+m+n=0$, $l^2+m^2-n^2=0$.
\item If a variable line in two adjacent positions has directions cosines $l, m, n$ and $l+\delta l, m+\delta m, n+\delta n$, show that the small angle $\delta\theta$ between the two positions is given by $$\delta\theta^2=\delta l^2+\delta m^2+\delta n^2$$ 
\item O is the origin and A is $(a,b,c)$.Find the direction cosines of the line OA and the equation of plane through A at right angle at OA.
\item Two systems of rectangular axis have the same origin. If a plane cuts them at distances $a,b,c$ and $a',b',c'$, respectively, from the origin, prove that $$\frac{1}{a^2}+\frac{1}{b^2}+\frac{1}{c^2}=\frac{1}{a'^2}+\frac{1}{b'^2}+\frac{1}{c'^2}$$.
\section*{Long Answer(L.A)}
\item Find the foot of perpendicular from the point $(2,3,-8)$ to the line \(\displaystyle \frac{4-x}{2}=\frac{y}{6}=\frac{1-z}{3}\).Also, find the perpendicular distance from the given point to the line.
\item Find the distance of a point $(2,4,-1)$ from the line $$\frac{x+5}{1}=\frac{y+3}{4}=\frac{z-6}{-9}$$
\item Find the length and the foot of perpendicular from the point $\left(  1, \displaystyle \frac{3}{2}\ ,2  \right)$ to the plane $2x-2y+4z+5=0.$
\item Find the equations of the line passing through the point $(3,0,1)$ and parallel to the planes $x+2y=0$ and $3y-z=0.$
\item Find the equation of the plane through the points $(2,1,-1)$ and $(-1,3,4),$ and 
perpendicular to the plane $x-2y+4z=10.$
\item Find the shortest distance between the lines given by $\overrightarrow{r}=(8+3\lambda\hat{i}-(9+16\lambda)\hat{j}+(10+7\lambda)\hat{k}$ and $\overrightarrow{r}=15\hat{i}+29\hat{j}+5\hat{k}+\mu(3\hat{i}+8\hat{j}-5\hat{k}).$
\item Find the equation of the plane which is perpendicular to the plane $5x+3y+6z+8=0$ and which contains the line of intersection of the planes $x+2y+3z-4=0$ and $2x+y-z+5=0.$
\item The plane $ax+by=0$ is rotated about its line of intersection with the plane $z=0$ through an angle $\alpha.$ Prove that the equation of the plane in its new position is $ax+by \pm (\sqrt{a^2+b^2} tan\alpha)z=0.$
\item Find the equation of the plane through the intersection of the planes $\overrightarrow{r} \cdot (\hat{i}+3\hat{j}) - 6=0$ and $\overrightarrow{r} \cdot (3\hat{i}-\hat{j}-4\hat{k})=0,$ whose perpendicular distance from origin is unity.
\item Show that the points $(\hat{i}-\hat{j}+3\hat{k})$ and $3(\hat{i}+\hat{j}+\hat{k})$ are equidistant from the plane $\overrightarrow{r} \cdot (5\hat{i}+2\hat{j}-7\hat{k})+9=0$ and lies on opposite side of it.
\item $\overrightarrow{AB}=3\hat{i}-\hat{j}+\hat{k}$ and $\overrightarrow{CD}=-3\hat{i}+2\hat{j}+4\hat{k}$ are two vectors. The position vectors of the points A and C are $6\hat{i}+7\hat{j}+4\hat{k}$ and $-9\hat{j}+2\hat{k},$ respectively. Find the position vector of a point P on the line AB and a point Q on the line CD such that $\overrightarrow{PQ}$ is perpendicular to $\overrightarrow{AB}$ and $\overrightarrow{CD}$ both.
\item Show that the straight lines whose direction cosines are given by $2l+2m-n=0$ and $mn+nl+lm=0$ are at right angles.
\item If $l_1, m_1, n_1;l_2, m_2, n_2;l_3, m_3, n_3$ are the direction cosines of the three mutually perpendcular lines, prove that the line whose direction cosines are propotional to $l_1+l_2+l_3 , m_1+m_2,m_3, n_1+n_2+n_3$ make angles with them.

\section*{Objective Type Questions}
Choose the correct answer from the given four options in each of the Exercises from 29 to 36.

\item Distance of the point $(\alpha \beta \gamma)$ from y-axis is
\begin{tasks}(4)
	\task $\beta$ 
	\task $|\beta|$
	\task $|\beta|+|\gamma|$
	\task $\sqrt{\alpha^2+\gamma^2}$
\end{tasks}
\item If the directions cosines of a line are $k,k,k,$ then
\begin{tasks}(4)
	\task $k>0$
	\task $0<k<1$
	\task $k=1$
	\task k=\(\displaystyle \frac{1}{\sqrt{3}}\) or -\(\displaystyle \frac{1}{\sqrt{3}}\)
\end{tasks}
\item The distance of the plane $\overrightarrow{r} \cdot \Biggl(\displaystyle \frac{2}{7}\hat{i}+\frac{3}{7}\hat{j}-\frac{6}{7}\hat{k}\ \Biggr)=1$ from the origin is 
\begin{tasks}(4)
	\task 1
	\task 7
	\task \(\displaystyle \frac{1}{7}\)
	\task None of these	
\end{tasks}
\item The sine of the angle between the straight line \(\displaystyle \frac{x-2}{3}=\frac{y-3}{4}=\frac{z-4}{5}\) and the plane $2x-2y+z=5$ is
\begin{tasks}(4)
	\task \(\displaystyle \frac{10}{6\sqrt{5}}\)
	\task \(\displaystyle \frac{4}{5\sqrt{2}}\)
	\task \(\displaystyle \frac{2\sqrt{3}}{5}\)
	\task \(\displaystyle \frac{\sqrt{2}}{10}\)

\end{tasks}
\item The reflection of the point $(\alpha \beta \gamma )$ in the xy-plane is 
\begin{tasks}(4)
	\task $\alpha,\beta,0)$
	\task $(0,0,\gamma)$
	\task $(-\alpha,-\beta,\gamma)$
	\task $(\alpha,\beta,-\gamma)$
\end{tasks}
\item The area of the quadrilateral ABCD, where A$(0,4,1)$, B$(2,3,-1)$, C$(4,5,0)$ and D$(2,6,2)$, is equal to 
\begin{tasks}(4)
	\task 9 sq. units
	\task 18 sq. units 
	\task 27 sq. units 
	\task 81 sq. units
\end{tasks}
\item The locus represented by $xy+yz=0$ is 
\begin{tasks}(2)
	\task A pair of perpendicular lines
	\task A pair of parallel lines
	\task A pair of parallel planes 
	\task A pair of perpendicular planes
\end{tasks}
\item The plane $2x-3y+6z-11=0$ makes an angle $sin^{-1}(\alpha)$ with x-axis. The value of $\alpha$ is equal to 
\begin{tasks}(4)
	\task \(\displaystyle \frac{\sqrt{3}}{2}\)
	\task \(\displaystyle \frac{\sqrt{2}}{3}\)
	\task \(\displaystyle \frac{2}{7}\)
	\task \(\displaystyle \frac{3}{7}\)
\end{tasks}



Fill in the blanks in each of the Exercises 37 to 41.
\item A plane passes through the points $(2,0,0) (0,3,0)$ and $(0,0,4)$.The equation of plane is \noindent\rule{2cm}{0.4pt}.
\item The direction cosines of the vector $(2\hat{i}+2\hat{j}-\hat{k})$ are \noindent\rule{2cm}{0.4pt}.
\item The vector equation of the line \(\displaystyle \frac{x-5}{3}=\frac{y+4}{7}=\frac{z-6}{2}\) is \noindent\rule{2cm}{0.4pt}. 
\item The vector equation of the line through the points $(3,4,-7)$ and $(1,-1,6)$ is \noindent\rule{2cm}{0.4pt}.
\item The cartesian equation of the plane $\overrightarrow{r} \cdot (\hat{i}+\hat{j}-\hat{k})=2$ is \noindent\rule{2cm}{0.4pt}.
\\\\State \textbf{True} or \textbf{False} for the statements in each of the Exercises 42 to 49.
\item the unit vector normal to the plane $x+2y+3z-6=0$ is \(\displaystyle \frac{1}{\sqrt{14}}\hat{i} + \frac{2}{\sqrt{14}}\hat{j} + \frac{3}{\sqrt{14}}\hat{k}\).
\item The intercepts made by the plane $2x-3y+5z+4=0$ on the co-ordinate axis are -2,\(\displaystyle \frac{4}{3},-\frac{4}{5}\).
\item The angle between the line $\overrightarrow{r}=(5\hat{i}-\hat{j}-4\hat{k})+\lambda(2\hat{i}-\hat{j}+\hat{k})$ and the plane $\overrightarrow{r} \cdot (3\hat{i}-4\hat{j}-\hat{k})+5=0$ is $sin^{-1}\Biggl(\displaystyle\frac{5}{2\sqrt{91}}\ \Biggr)$.
\item The angle between the planes $\overrightarrow{r} \cdot (2\hat{i}-3\hat{j}+\hat{k})=1$ and $\overrightarrow{r} \cdot (\hat{i}-\hat{j})=4$ is $cos^{-1} \Biggl(\displaystyle\frac{-5}{\sqrt{58}}\ \Biggr)$.
\item The line $\overrightarrow{r}=2\hat{i}-3\hat{j}-\hat{k}+\lambda(\hat{i}-\hat{j}+2\hat{k})$ lies in the plane $\overrightarrow{r} \cdot (3\hat{i}+\hat{j}-\hat{k})+2=0$.
\item The vector equation of the line \(\displaystyle\frac{x-5}{3}=\frac{y+4}{7}=\frac{z-6}{2}\) is\\
$$\overrightarrow{r}=5\hat{i}-4\hat{j}+6\hat{k}+\lambda(3\hat{i}+7\hat{j}+2\hat{k}).$$
\item The equation of a line, which is parallel to $2\hat{i}+\hat{j}+3\hat{k}$ and which passes through the point $(5,-2,4)$ is \(\displaystyle \frac{x-5}{2}=\frac{y+2}{-1}=\frac{z-4}{3}\).
\item If the foot of perpendicular drawn from the origin to a plane is $(5,-3,-2)$, then the equation of plane is $\overrightarrow{r} \cdot (5\hat{i}-3\hat{j}-2\hat{k})=38.$
\end{enumerate}
\end{document}
